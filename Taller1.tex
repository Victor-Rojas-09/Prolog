\documentclass[12pt]{article}
\usepackage[utf8]{inputenc}
\usepackage[spanish]{babel}
\usepackage{graphicx}
\usepackage{geometry}
\geometry{a4paper, margin=1in}

\title{\textbf{Taller 1:} \\[0.3cm] Lógica de Predicados y Prolog}
\author{Víctor Manuel Rojas \\ Isabella Pelaez Duque}
\date{}

\begin{document}

\maketitle

\section*{Ejercicio 1:}
De acuerdo a la siguiente imagen de árbol genealógico, construya una lógica de predicados donde las relaciones directas se generen por hechos y las relación de más de una generación se obtengan mediante reglas. \\[0.3cm]
Ejemplo: \\ 
\textit{padre(homero,bart)} el anterior es un hecho y corresponde a una relación directa, mientras que \\ 
\textit{abuelo(X,bart)} puede ser una consulta hecha al programa, donde la relación abuelo (que es de más de 1 generación) debe ser obtenida por reglas, no por hechos.

\\[0.5cm]
\begin{center}
    \includegraphics[width=0.8\textwidth]{FamilyTree.png}
\end{center}

\section*{Ejercicio 2:}
La ley dice que es un crimen para un Estadounidense vender armas a naciones hostiles. Corea del Sur, enemigo de Estados Unidos, tiene algunos misiles, y todos sus misiles les fueron vendidos por el Coronel West, quien es Estadounidense. \\[0.3cm]
Pruebe que el Col. West es un criminal.

\end{document}
